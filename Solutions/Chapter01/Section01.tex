%!TEX root = ../../Algorithms.tex
\documentclass[Chapter01]{subfiles}

\begin{document}
	\subsection{Algorithms}

	\begin{enumerate}[leftmargin=\labelsep]
		% 1.1-1
		\item Give a real-world example that requires sorting or a real-world example that requires computing a convex hull.
		\begin{answer}
			A company might keep a database of all customers who have purchased products or services from them. At the end of the year, this company might offer a certain sale to its customers, but can only afford to offer a certain number of these sales. They would prefer to offer them to their best customers first, but if those customers aren't interested in accepting the deal, they would offer it to the next best customer. This would require sorting the customers based on how much money they have spent.\\
			\hfill
			A game theorist formulates a real-world problem (such as bidding for project against rival companies) as a game. The company would like to know whether a particular payoff is possible if they have a particular mixed strategy. The game theorist knows that the possible payoffs for this game when using mixed strategies is the convex hull of the payoffs possible when only using non-mixed strategies, which are known. They would simply compute this convex hull, and check whether or not the given payout is contained within.
		\end{answer}

		% 1.1-2
		\item Other than speed, what other measures of efficiency might one use in a real-world setting?
		\begin{answer}
			Storage space required and power used could be useful measures.
		\end{answer}

		% 1.1-3
		\item Select a data structure that you have seen previously, and discuss its strengths and limitations.
		\begin{answer}
			Linked lists are useful when the length of the list changes often, as new links can be inserted and removed without re-allocating every other link. However, it can take a long time to reach a particular link, as other links may have to be traversed to reach it. As well, the construction requires storing elements in the heap (instead of the stack), which is slower to access.
		\end{answer}

		% 1.1-4
		\item How are the shortest-path and traveling-salesman problems given above similar? How are they different?
		\begin{answer}
			Both problems require the use of a ``road map'' and minimize the distance traveled in completing some task. In the shortest-path problem, we don't require that every destination is visited (and, in fact, they probably aren't), but we do require this of the traveling salesman problem. This means that adding additional roads and destinations to the map might not change the shortest path (and it won't, unless this introduces a shortcut), but will always change the solution to the traveling salesman problem.
		\end{answer}

		% 1.1-5
		\item Come up with a real-world problem in which only the best solution will do. Then come up with one in which a solution that is ``approximately'' the best is good enough.
		\begin{answer}
			At the end of an Olympic event, a list of competitors in that event each have a certain number of points. Medals are given to the three competitors who have more points then anyone else. The problem of finding these three competitors is exact - if someone did better than everyone else and did not get a medal, the problem would not be solved appropriately.\\
			\hfill
			If we would analyze the traveling salesman problem talked about in the chapter, however, we would notice that the exact solution isn't needed. It's worth finding an approximate solution, as the amount of distance saved over an entire route could be tens or hundreds of miles, but the difference between an approximate solution and the exact solution would only be a few miles (the cost of which being negligible).
		\end{answer}
	\end{enumerate}

\end{document}