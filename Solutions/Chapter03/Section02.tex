%!TEX root = ../../Algorithms.tex
\documentclass[Chapter03]{subfiles}

\begin{document}
	\subsection{Standard notations and common functions}

	\begin{enumerate}
		% 3.2-1
		\item Show that if $f(n)$ and $g(n)$ are monotonically increasing functions, then so are the functions $f(n) + g(n)$ and $f(g(n))$, and if $f(n)$ and $g(n)$ are in addition nonnegative, then $f(n)g(n)$ is monotonically increasing.
		\begin{answer}
			Let $f,g$ be monotonically increasing functions and $a,b \in \ints$ such that $a < b$. Then $f(a) \leq f(b)$ and $g(a) \leq g(b)$. Then $f(a) + g(a) \leq f(b) + g(b)$ by adding the previous two inequalities together, so $f(n) + g(n)$ is monotonically increasing. As well, since $g(a) \leq g(b)$ and $f$ is monotonically increasing, then $f(g(a)) \leq f(g(b))$, so $f(g(n))$ is monotonically increasing. Finally, if $g(a),f(b) \geq 0$,  then $f(a)g(a) \leq f(b)g(a) \leq f(b)g(b)$, so $f(n)g(n)$ is monotonically increasing.
		\end{answer}

		% 3.2-2
		\item Prove \cref{eq:ch03-book-16}.
		\begin{align}
			a^{\log_b(c)} = c^{\log_b(a)} \label{eq:ch03-book-16} \tag{3.16}
		\end{align}
		\begin{answer}
			We have
			\begin{align*}
				a^{\log_b(c)} &= a^{\frac{\log_a(c)}{\log_a(b)}}\\
					&= \qty(a^{\log_a(c)})^{\frac{1}{\log_a(b)}}\\
					&= \cancelto{c}{\qty(a^{\log_a(c)})}^{\frac{1}{\log_a(b)}}\\
					&= c^{\frac{\log_b(a)}{\log_b(b)}}\\
					&= c^{\frac{\log_b(a)}{\cancel{\log_b(b)}}}\\
					&= c^{\log_b(a)}.
			\end{align*}
		\end{answer}

		% 3.2-3 TODO
		\item Prove \cref{eq:ch03-book-19}. Also prove that $n! = \omega(2^n)$ and $n! = o(n^n)$.
		\begin{align}
			\lg(n!) = \Theta(n \lg(n)) \label{eq:ch03-book-19} \tag{3.19}
		\end{align}
		\begin{answer}
			By properties of logarithms, we have
			\begin{align*}
				\lg(n!) &= \lg \qty(\prod_{i = 1}^n i)\\
					&= \sum_{i = 1}^n \lg(i)\\
					&= \Theta(n \lg(n)),
			\end{align*}
			which follows from \cref{exer:appA-log-sum-bound}.

			% The following proof from https://math.stackexchange.com/q/640609
			% It's a toughie - most people use this "Gauss's trick" - but leave the tail end of the proof to be observed
			% This method instead leaves no room for error
			We would like to show that $n! \geq n^{n/2}$ or alternatively $(n!)^2 \geq n^n$. We can show this with induction, using this fact:
			\begin{align}
				(1 + x)^n \geq 1 + xn,\quad x \geq 0 \label[inequality]{eq:ch03-bernoulli-inequality}
			\end{align}

			  (which we will also prove) and noting that $(1!)^2 = 1 \geq 1 = 1^1$, $(1 + x)^1 \geq 1 + 1x$, and $(1 + x)^2 = 1 + 2x + x^2 \geq 1 + 2x$. Then for some $n \geq 1$ assume that $(n!)^2 \geq n^n$ and $(1 + x)^n \geq 1 + xn$ for all $x \geq 0$. Then we have
			\begin{align*}
				((n + 1)!)^2 &= (n!)^2 (n+1)^2\\
					&\geq n^n (n + 1)^2\\
					&= n n^{n - 1}(n + 1)^2\\
					&= n \qty(\frac{n - 1}{n})^{n - 1} \qty(\frac{n}{n - 1})^{n - 1} n^{n - 1}(n + 1)^2\\
					&= n \qty(1 - \frac{1}{n})^{n - 1} \frac{n^{2(n - 1)}(n + 1)^2}{(n - 1)^{n - 1}}\\
					&\overset{\mathclap{\eqref{eq:ch03-bernoulli-inequality}}}{\geq} \cancel{n \qty(1 - \frac{n - 1}{n})} \frac{n^{2(n - 1)}(n + 1)^2}{(n - 1)^{n - 1}}\\
				\shortintertext{(with $x = -\frac{1}{n}$)}
					&= (1) \frac{n^{2(n - 1)}(n + 1)^2}{(n - 1)^{n - 1}}\\
					&\geq \qty (1 - \frac{1}{n^2})^{n - 1} \frac{n^{2(n - 1)}(n + 1)^2}{(n - 1)^{n - 1}}\\
				\shortintertext{(since $0 < \frac{1}{n^2} \leq 1$, then $0 \leq (1 - \frac{1}{n^2})^{n - 1} < 1$)}
					&= \qty(\frac{n^2 - 1}{\cancel{n^2}})^{n - 1} \frac{\cancel{n^{2(n - 1)}}(n + 1)^2}{(n - 1)^{n - 1}}\\
					&= (\cancel{(n-1)}(n+1))^{n - 1} \frac{(n + 1)^2}{\cancel{(n - 1)^{n - 1}}}\\
					&= (n + 1)^{n + 1},
			\end{align*}
			\begin{align*}
				(1 + x)^{n + 2} &= (1 + x)^n(1 + x)^2\\
					&\geq (1 + xn) (1 + 2x + x^2)\\
					&= (1 + 2x + x^2) + xn + 2x^2n + x^3n\\
					&= 1 + (n + 2)x + (2n + 1)x^2 + nx^3\\
					&\geq 1 + (n + 2)x
			\end{align*}
			Then by induction $(n!)^2 \geq n^n$ and we have 
			\begin{align*}
				n! > c2^n \tag*{$\impliedby$}\\
				n^{n/2} > c2^n \tag*{$\impliedby$}\\
				16^{n/2} > c2^n \tag*{$\impliedby$}\\
				\shortintertext{(if $n > 16$)}
				4^n > c2^n \tag*{$\impliedby$}\\
				2^n > c \tag*{$\impliedby$}\\
				n > \lg(c),
			\end{align*}
			so $n! = \omega(2^n)$.
		\end{answer}

		% 3.2-4
		\item $\bigstar$ Is the function $\lceil \lg(n) \rceil !$ polynomially bounded? Is the function $\lceil \lg(\lg(n)) \rceil !$ polynomially bounded?
		\begin{answer}
		% 	A function $f(n)$ is polynomially bounded if there exist $k,c_1,n_0 > 0$ such that $f(n) \leq c_1n^k$ whenever $n > n_0$, but this is true if and only if $\lg(f(n)) \leq \lg(c_1n^k) = c_2k\lg(n)$, i.e. $\lg(f(n)) = \bigO(\lg(n))$.

		% 	We have
		% 	\begin{align*}
		% 		\lg(\lceil \lg(n) \rceil !) &= \Theta(\lceil \lg(n) \rceil \lg(\lceil \lg(n) \rceil))\\
		% 			&= \Theta(\lg(n) \lg(\lg(n)))\\
		% 			&= \omega(\lg(n)),
		% 	\end{align*}
		% 	which means
		\end{answer}

		% 3.2-5
		\item $\bigstar$ Which is asymptotically larger: $\lg(\lg^*(n))$ or $\lg^*(\lg(n))$?
		\begin{answer}
			Since $\lg^*(\lg(n)) = \lg^*(n) - 1$ (we just pre-apply a single $\lg$ before counting the number of $\lg$), and $\lg(n) = o(n - 1)$, then $\lg(\lg^*(n)) = o(\lg^*(n) - 1)$.
		\end{answer}

		% 3.2-6
		\item Show that the golden ratio $\varphi$ and its conjugate $\hat\varphi$ both satisfy the equation $x^2 = x + 1$.
		\begin{answer}
			
		\end{answer}

		% 3.2-7
		\item Prove by induction that the $i$-th Fibonacci number satisfies the equality
		\[
			F_i = \frac{1}{\sqrt{5}}(\varphi^i - \hat\varphi^i),
		\]
		where $\varphi$ is the golden ratio and $\hat\varphi$ is its conjugate.
		\begin{answer}
			
		\end{answer}

		% 3.2-8
		\item Show that $k\ln(k) = \Theta(n)$ implies $k = \Theta(n / \ln(n))$.
		\begin{answer}
			
		\end{answer}

	\end{enumerate}

\end{document}