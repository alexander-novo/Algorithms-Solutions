%!TEX root = ../../Algorithms.tex
\documentclass[Chapter02]{subfiles}

\tikzset{
	node distance = .5cm,
	merge node/.style = {
		on chain
	},
	merge box/.style = {
		draw,
		rectangle,
		fill = gray
	},
	every path/.style = {
		-{Stealth},
		very thick
	},
	level 1/.style = {
		shorten >= .5cm,
		shorten <= .5cm
	},
	level 2/.style = {
		shorten >= .25cm,
		shorten <= .25cm
	},
	level 3/.style = {
		shorten >= .15cm,
		shorten <= .15cm
	}
}

\begin{document}
	\subsection{Designing algorithms}

	\begin{enumerate}[leftmargin=\labelsep]
		%2.3-1
		\item Using Figure 2.4 as a model, illustrate the operation of merge sort on the array $A = \langle 3, 41, 52, 26, 38, 57, 9, 49 \rangle$.

		\begin{figure}[H]
			\centering
			\tikzsetnextfilename{Chapter02/figure-2.4}
			\begin{tikzpicture}
				\begin{scope}[start chain = A going right]
					\node[merge node] {1};
					\node[merge node] {2};
					\node[merge node] {2};
					\node[merge node] {3};
					\node[merge node] {4};
					\node[merge node] {5};
					\node[merge node] {6};
					\node[merge node] {7};

					\begin{scope}[on background layer]
						\node[merge box, fit = (A-1) (A-8), label={sorted sequence}] (box1) {};
					\end{scope}
				\end{scope}

				\begin{scope}[start chain = A going left]
					\node[merge node, below left = 1.4 and 2.25 of box1.south] {7};
					\node[merge node] {5};
					\node[merge node] {4};
					\node[merge node] {2};

					\begin{scope}[on background layer]
						\node[merge box, fit = (A-1) (A-4)] (box2-1) {};
					\end{scope}
				\end{scope}

				\begin{scope}[start chain = A going left]
					\node[merge node, below left = 1.4 and 1.25 of box2-1.south] {5};
					\node[merge node] {2};

					\begin{scope}[on background layer]
						\node[merge box, fit = (A-1) (A-2)] (box3-1) {};
					\end{scope}
				\end{scope}

				\node[merge box, below left = 1.4 and .75 of box3-1.south] (box4-1) {5};
				\node[merge box, below right = 1.4 and .75 of box3-1.south] (box4-2) {2};

				\begin{scope}[start chain = A going right]
					\node[merge node, below right = 1.4 and 1.25 of box2-1.south] {4};
					\node[merge node] {7};

					\begin{scope}[on background layer]
						\node[merge box, fit = (A-1) (A-2)] (box3-2) {};
					\end{scope}
				\end{scope}

				\node[merge box, below left = 1.4 and .75 of box3-2.south] (box4-3) {4};
				\node[merge box, below right = 1.4 and .75 of box3-2.south] (box4-4) {7};

				\begin{scope}[start chain = A going right]
					\node[merge node, below right = 1.4 and 2.25 of box1.south] {1};
					\node[merge node] {2};
					\node[merge node] {3};
					\node[merge node] {6};

					\begin{scope}[on background layer]
						\node[merge box, fit = (A-1) (A-4)] (box2-2) {};
					\end{scope}
				\end{scope}

				\begin{scope}[start chain = A going left]
					\node[merge node, below left = 1.4 and 1.25 of box2-2.south] {3};
					\node[merge node] {1};

					\begin{scope}[on background layer]
						\node[merge box, fit = (A-1) (A-2)] (box3-3) {};
					\end{scope}
				\end{scope}

				\node[merge box, below left = 1.4 and .75 of box3-3.south] (box4-5) {1};
				\node[merge box, below right = 1.4 and .75 of box3-3.south] (box4-6) {3};

				\begin{scope}[start chain = A going right]
					\node[merge node, below right = 1.4 and 1.25 of box2-2.south] {2};
					\node[merge node] {6};

					\begin{scope}[on background layer]
						\node[merge box, fit = (A-1) (A-2)] (box3-4) {};
					\end{scope}
				\end{scope}

				\node[merge box, below left = 1.4 and .75 of box3-4.south] (box4-7) {2};
				\node[merge box, below right = 1.4 and .75 of box3-4.south] (box4-8) {6};

				% ==============================================================================

				\node[below = .7 of box1] {merge};

				\node[below = .7 of box2-1] {merge};
				\node[below = .7 of box2-2] {merge};

				\node[below = .7 of box3-1] {merge};
				\node[below = .7 of box3-2] {merge};
				\node[below = .7 of box3-3] {merge};
				\node[below = .7 of box3-4] {merge};

				\node[fit = (box4-1) (box4-8)] (bottom) {};
				\node[below = 0 of bottom] {initial sequence};

				% ==============================================================================


				\draw[level 1] (box2-1.north) -- (box1.200);
				\draw[level 1] (box2-2.north) -- (box1.340);

				\draw[level 2] (box3-1.north) -- (box2-1.220);
				\draw[level 2] (box3-2.north) -- (box2-1.320);
				\draw[level 2] (box3-3.north) -- (box2-2.220);
				\draw[level 2] (box3-4.north) -- (box2-2.320);

				\draw[level 3] (box4-1.north) -- (box3-1.240);
				\draw[level 3] (box4-2.north) -- (box3-1.300);
				\draw[level 3] (box4-3.north) -- (box3-2.240);
				\draw[level 3] (box4-4.north) -- (box3-2.300);
				\draw[level 3] (box4-5.north) -- (box3-3.240);
				\draw[level 3] (box4-6.north) -- (box3-3.300);
				\draw[level 3] (box4-7.north) -- (box3-4.240);
				\draw[level 3] (box4-8.north) -- (box3-4.300);
			\end{tikzpicture}

			\caption{Figure 2.4 from the book, included for reference.}
		\end{figure}
		\begin{answer}

			\begin{figure}[H]
				\centering
				\tikzsetnextfilename{Chapter02/merge-answer}
				\begin{tikzpicture}
					\begin{scope}[start chain = A going right]
						\node[merge node] {3};
						\node[merge node] {9};
						\node[merge node] {26};
						\node[merge node] {38};
						\node[merge node] {41};
						\node[merge node] {49};
						\node[merge node] {52};
						\node[merge node] {57};

						\begin{scope}[on background layer]
							\node[merge box, fit = (A-1) (A-8), label={sorted sequence}] (box1) {};
						\end{scope}
					\end{scope}

					\begin{scope}[start chain = A going left]
						\node[merge node, below left = 1.4 and 2.25 of box1.south] {52};
						\node[merge node] {41};
						\node[merge node] {26};
						\node[merge node] {3};

						\begin{scope}[on background layer]
							\node[merge box, fit = (A-1) (A-4)] (box2-1) {};
						\end{scope}
					\end{scope}

					\begin{scope}[start chain = A going left]
						\node[merge node, below left = 1.4 and 1.25 of box2-1.south] {41};
						\node[merge node] {3};

						\begin{scope}[on background layer]
							\node[merge box, fit = (A-1) (A-2)] (box3-1) {};
						\end{scope}
					\end{scope}

					\node[merge box, below left = 1.4 and .75 of box3-1.south] (box4-1) {3};
					\node[merge box, below right = 1.4 and .75 of box3-1.south] (box4-2) {41};

					\begin{scope}[start chain = A going right]
						\node[merge node, below right = 1.4 and 1.25 of box2-1.south] {26};
						\node[merge node] {52};

						\begin{scope}[on background layer]
							\node[merge box, fit = (A-1) (A-2)] (box3-2) {};
						\end{scope}
					\end{scope}

					\node[merge box, below left = 1.4 and .75 of box3-2.south] (box4-3) {52};
					\node[merge box, below right = 1.4 and .75 of box3-2.south] (box4-4) {26};

					\begin{scope}[start chain = A going right]
						\node[merge node, below right = 1.4 and 2.25 of box1.south] {9};
						\node[merge node] {38};
						\node[merge node] {49};
						\node[merge node] {57};

						\begin{scope}[on background layer]
							\node[merge box, fit = (A-1) (A-4)] (box2-2) {};
						\end{scope}
					\end{scope}

					\begin{scope}[start chain = A going left]
						\node[merge node, below left = 1.4 and 1.25 of box2-2.south] {57};
						\node[merge node] {38};

						\begin{scope}[on background layer]
							\node[merge box, fit = (A-1) (A-2)] (box3-3) {};
						\end{scope}
					\end{scope}

					\node[merge box, below left = 1.4 and .75 of box3-3.south] (box4-5) {38};
					\node[merge box, below right = 1.4 and .75 of box3-3.south] (box4-6) {57};

					\begin{scope}[start chain = A going right]
						\node[merge node, below right = 1.4 and 1.25 of box2-2.south] {9};
						\node[merge node] {49};

						\begin{scope}[on background layer]
							\node[merge box, fit = (A-1) (A-2)] (box3-4) {};
						\end{scope}
					\end{scope}

					\node[merge box, below left = 1.4 and .75 of box3-4.south] (box4-7) {9};
					\node[merge box, below right = 1.4 and .75 of box3-4.south] (box4-8) {49};

					% ==============================================================================

					\node[below = .7 of box1] {merge};

					\node[below = .7 of box2-1] {merge};
					\node[below = .7 of box2-2] {merge};

					\node[below = .7 of box3-1] {merge};
					\node[below = .7 of box3-2] {merge};
					\node[below = .7 of box3-3] {merge};
					\node[below = .7 of box3-4] {merge};

					\node[fit = (box4-1) (box4-8)] (bottom) {};
					\node[below = 0 of bottom] {initial sequence};

					% ==============================================================================


					\draw[level 1] (box2-1.north) -- (box1.200);
					\draw[level 1] (box2-2.north) -- (box1.340);

					\draw[level 2] (box3-1.north) -- (box2-1.220);
					\draw[level 2] (box3-2.north) -- (box2-1.320);
					\draw[level 2] (box3-3.north) -- (box2-2.220);
					\draw[level 2] (box3-4.north) -- (box2-2.320);

					\draw[level 3] (box4-1.north) -- (box3-1.240);
					\draw[level 3] (box4-2.north) -- (box3-1.300);
					\draw[level 3] (box4-3.north) -- (box3-2.240);
					\draw[level 3] (box4-4.north) -- (box3-2.300);
					\draw[level 3] (box4-5.north) -- (box3-3.240);
					\draw[level 3] (box4-6.north) -- (box3-3.300);
					\draw[level 3] (box4-7.north) -- (box3-4.240);
					\draw[level 3] (box4-8.north) -- (box3-4.300);
				\end{tikzpicture}

				\caption{Merging $A$.}
				\label{fig:ch02-merge}
			\end{figure}

			See \autoref{fig:ch02-merge}.
		\end{answer}

		%2.3-2
		\item Rewrite the \textsc{Merge} procedure so that it does not use sentinels, instead stopping once either array $L$ or $R$ has had all its elements copied back to $A$ and then copying the remained of the other array back into $A$.
		\begin{answer}
			See \autoref{alg:ch02-Merge}.

			\begin{algorithm}[H]
				\caption{\textsc{Merge} redesigned to not use sentinels.}
				\label{alg:ch02-Merge}
				\SetKwProg{Fn}{def}{\string:}{end}
				\SetKwFunction{FMerge}{Merge}

				\Fn{\FMerge{$A[p \dots q, q + 1, \dots \dots r]$}}{
					$n_1 \leftarrow q - p + 1$\;
					$n_2 \leftarrow r - q$\;
					\For{$i = 1\ \KwTo\ n_1$}{
						$L[i] \leftarrow A[p + i - 1]$\;
					}
					\For{$j = 1\ \KwTo\ n_2$}{
						$R[j] \leftarrow A[q + j]$\;
					}
					$i \leftarrow 1$\;
					$j \leftarrow 1$\;
					\For{$k = 1\ \KwTo\ \min\{n_1, n_2\}$}{
						\eIf{$L[i] \leq R[j]$}{
							$A[k] \leftarrow L[i]$\;
							$i \leftarrow i + 1$\;
						}{
							$A[k] \leftarrow R[j]$\;
							$j \leftarrow j + 1$\;
						}
					}
					\For{$i\ \KwTo\ n_1$}{
						$A[k] \leftarrow L[i]$\;
						$k \leftarrow k + 1$\;
					}
					\For{$j\ \KwTo\ n_2$}{
						$A[k] \leftarrow R[j]$\;
						$k \leftarrow k + 1$\;
					}
				}
			\end{algorithm}
		\end{answer}

		%2.3-3
		\item Use mathematical induction to show that when $n$ is an exact power of 2, the solution of the recurrence
		\[
			T(n) = 
			\begin{cases}
				2, & n = 2\\
				2T(n / 2) + n, & n = 2^k,\ k > 1
			\end{cases}
		\]
		is $T(n) = n\lg(n)$.
		\begin{answer}
			For $n = 2^1$, we have $T(2) = 2 = 2 \cdot 1 = 2 \lg(2)$, so the solution works for $k = 1$. Then, assume $T(2^k) = 2^k \lg(2^k)$ for some $k$. We have
			\begin{align*}
				T(2^{k + 1}) &= 2T(2^{k+1} / 2) + 2^{k + 1}\\
					&=  2T(2^k) + 2^{k + 1}\\
					&= 2\qty(2^k \lg(2^k)) + 2^{k + 1}\\
					&= 2^{k + 1} \lg(2^k) + 2^{k + 1}\\
					&= 2^{k + 1}(\lg(2^k) + 1)\\
					&= 2^{k + 1}(\lg(2^k) + \lg(2))\\
					&= 2^{k + 1}\lg(2 \cdot 2^k)\\
					&= 2^{k + 1}\lg(2^{k + 1}).
			\end{align*}
			Threfore, by induction, the solution works for all $k \geq 1$.
		\end{answer}

		%2.3-4
		\item We can express insertion sort as a recursive procedure as follows. In order to sort $A[1 \dots n]$, we recursively sort $A[1 \dots n-1]$ and then insert $A[n]$ into the sorted array $A[1 \dots n-1]$. Write a recurrence for the worst-case running time of this recursive version of insertion sort.
		\begin{answer}
			Since the worst-case running time of inserting an element into a sorted array of size $n$ is $n$ (as we might have to insert the element at the front of the sorted array), our recurrence is
			\begin{align*}
				T(n) &= T(n - 1) + (n - 1).
			\end{align*}
		\end{answer}

		%2.3-5
		\item \label{exer:ch02-binary-search} Referring back to the searching problem (see Exercise \ref{exer:ch02-linear-search}), observe that if the sequence $A$ is sorted, we can check the midpoint of the sequence against $v$ and eliminate half of the sequence from further consideration. The \textbf{\textit{binary search}} algorithm repeats this procedure, halving the size of the remaining portion of the sequence each time. Write pseudocode, either iterative or recursive, for \textsc{Binary-Search}. Argue that the worst-case running time of binary search is $\Theta(\lg(n))$.
		\begin{answer}
			See \autoref{alg:ch02-binary-search}.

			\begin{algorithm}[H]
				\caption{\textsc{Binary-Search}}
				\label{alg:ch02-binary-search}
				\SetKwProg{Fn}{def}{\string:}{end}
				\SetKwFunction{FSearch}{Binary-Search}

				\Fn{\FSearch{$A[a \dots b], v$}}{
					\If{$b < a$}{
						\Return{{\tt NIL}}\;
					}

					$m \leftarrow \left\lfloor \frac{a + b}{2} \right\rfloor$\;
					\uIf{$A[m] = v$}{
						\Return{m}\;
					}
					\uElseIf{$v > A[m]$}{
						\Return{\FSearch{$A[m + 1 \dots b], v$}}\;
					}
					\Else{
						\Return{\FSearch{$A[a \dots m - 1], v$}}\;
					}
				}
			\end{algorithm}

			The worst-case scenario for this algorithm happens when it terminates due to $b < a$ - this happens when there are only one or two elements left in $A$, and neither one of them are $v$. Since the algorithm effectively halves the size of $A$ each time it runs, we have a recurrence for the worst-case runtime $T(n)$ of the algorithm in the worst-case on an array of size $n$:
			\[
				T(n) =
				\begin{cases}
					c, &n \leq 2\\
					T(n/2) + c &n > 2
				\end{cases}.
			\]
			Then we have
			\begin{align*}
				T(n) &= \sum_{i = 1}^{\lg(n)} c\\
				     &= c\lg(n)\\
				     &= \Theta(\lg(n)).
			\end{align*}
		\end{answer}

		%2.3-6
		\item Observe that the \textbf{while} loop of lines 5-7 of the \textsc{Insertion-Sort} procedure in Section 2.1 uses a linear search to scan (backward) through the sorted sub-array $A[1 \dots j - 1]$. Can we use binary search (see Exercise \ref{exer:ch02-binary-search}) instead to improve the overall worst-case running time of insertion sort to $\Theta(n \lg(n))$?
		\begin{answer}
			
		\end{answer}

		%2.3-7
		\item $\bigstar$ Describe a $\Theta(n \lg(n))$-time algorithm that, given a set $S$ of $n$ integers and another integer $x$, determines whether or not there exist two elements in $S$ whose sum is exactly $x$.
		\begin{answer}
			
		\end{answer}

		
	\end{enumerate}
\end{document}