%!TEX root = ../../Algorithms.tex
\documentclass[Chapter02]{subfiles}

\begin{document}
	\subsection{Insertion Sort}

	\begin{enumerate}[leftmargin=\labelsep]
		% 2.1-1
		\item Using Figure 2.2 as a model, illustrate the operation of \textsc{Insertion-Sort} on the array $A = \langle 31, 41, 59, 26, 41, 58 \rangle$.
		\begin{answer}
			
		\end{answer}

		% 2.1-2
		\item Rewrite the \textsc{Insertion-Sort} procedure to sort into nonincreasing instead of nondecreasing order.
		\begin{answer}
			
		\end{answer}

		% 2.1-3
		\item Consider the \textbf{\textit{searching problem}}:\\[.5em]
		\textbf{Input:} A sequence of $n$ number $A = \langle a_1, a_2, \dots, a_n \rangle$ and a value $v$.\\[.5em]
		\textbf{Output:} An index $i$ such that $v = A[i]$ or the special value \texttt{NIL} if $v$ does not appear in $A$.\\[.5em]
		Write pseudocode for \textbf{\textit{linear-search}}, which scans through the sequence, looking for $v$. using a loop invariant, prove that your algorithm is correct. Make sure that your loop invariant fulfills the three necessary properties.
		\begin{answer}
			
		\end{answer}

		% 2.1-4
		\item Consider the problem of adding two $n$-bit binary integers, stored in two $n$-element arrays $A$ and $B$, The sum of the two integers should be stored in binary form in an $(n + 1)$-element array $C$. State the problem formally and write pseudocode for adding the two integers.
		\begin{answer}
			
		\end{answer}
	\end{enumerate}

\end{document}