%!TEX root = ../../Algorithms.tex
\documentclass[Chapter02]{subfiles}

\tikzset{
	node distance = 0pt,
	sort node/.style = {
		rectangle,
		draw = black,
		on chain
	},
	curs node/.style = {
		sort node,
		text = white,
		fill = black
	},
	comp node/.style = {
		sort node,
		fill = gray
	},
	every label/.style = {font=\tiny}
}



\begin{document}
	\subsection{Insertion Sort}

	\begin{enumerate}[leftmargin=\labelsep]
		% 2.1-1
		\item Using Figure 2.2 as a model, illustrate the operation of \textsc{Insertion-Sort} on the array $A = \langle 31, 41, 59, 26, 41, 58 \rangle$.
		\begin{figure}[H]
			\floatsetup{valign=t, heightadjust=object, style=plaintop}
			\begin{subfloatrow}[3]
				\centering
				\floatbox{figure}{\caption{}}{
					\tikzsetnextfilename{Chapter02/insertion-sort-01}
					\scalebox{1.5}{
						\begin{tikzpicture}[start chain = A going right]
							\node [comp node, label=1] {5};
							\node [curs node, label=2] {2};
							\node [sort node, label=3] {4};
							\node [sort node, label=4] {6};
							\node [sort node, label=5] {1};
							\node [sort node, label=6] {3};

							\draw [-{Stealth[length=1mm]}, gray] (A-1.290) to[out=270, in=270, looseness=2] (A-2.250);
							\draw [-{Stealth[length=1mm]}]       (A-2.300) to[out=270, in=270, looseness=2] (A-1.240);
						\end{tikzpicture}
					}
				}
				\hfill
				\floatbox{figure}{\caption{}}{
					\tikzsetnextfilename{Chapter02/insertion-sort-02}
					\scalebox{1.5}{
						\begin{tikzpicture}[start chain = A going right]
							\node[comp node, label=1] {2};
							\node[comp node, label=2] {5};
							\node[curs node, label=3] {4};
							\node[sort node, label=4] {6};
							\node[sort node, label=5] {1};
							\node[sort node, label=6] {3};

							\draw [-{Stealth[length=1mm]}, gray] (A-2.290) to[out=270, in=270, looseness=2] (A-3.250);
							\draw [-{Stealth[length=1mm]}]       (A-3.300) to[out=270, in=270, looseness=2] (A-2.240);
						\end{tikzpicture}
					}
				}
				\hfill
				\floatbox{figure}{\caption{}}{
					\tikzsetnextfilename{Chapter02/insertion-sort-03}
					\scalebox{1.5}{
						\begin{tikzpicture}[start chain = A going right]
							\node[sort node, label=1] {2};
							\node[sort node, label=2] {4};
							\node[comp node, label=3] {5};
							\node[curs node, label=4] {6};
							\node[sort node, label=5] {1};
							\node[sort node, label=6] {3};

							\draw [-{Stealth[length=1mm]}] (A-4.300) to[out=270, in=270, looseness=4] (A-4.240);
						\end{tikzpicture}
					}
				}
			\end{subfloatrow}

			\begin{subfloatrow}[3]
				\centering
				\floatbox{figure}{\caption{}}{
					\tikzsetnextfilename{Chapter02/insertion-sort-04}
					\scalebox{1.5}{
						\begin{tikzpicture}[start chain = A going right]
							\node [comp node, label=1] {2};
							\node [comp node, label=2] {4};
							\node [comp node, label=3] {5};
							\node [comp node, label=4] {6};
							\node [curs node, label=5] {1};
							\node [sort node, label=6] {3};

							\draw [-{Stealth[length=1mm]}, gray] (A-1.290) to[out=270, in=270, looseness=2] (A-2.250);
							\draw [-{Stealth[length=1mm]}, gray] (A-2.290) to[out=270, in=270, looseness=2] (A-3.250);
							\draw [-{Stealth[length=1mm]}, gray] (A-3.290) to[out=270, in=270, looseness=2] (A-4.250);
							\draw [-{Stealth[length=1mm]}, gray] (A-4.290) to[out=270, in=270, looseness=2] (A-5.250);
							\draw [-{Stealth[length=1mm]}, rounded corners=.3cm]   (A-5.300) -- +(0,-.4) -| (A-1.240);
						\end{tikzpicture}
					}
				}
				\hfill
				\floatbox{figure}{\caption{}}{
					\tikzsetnextfilename{Chapter02/insertion-sort-05}
					\scalebox{1.5}{
						\begin{tikzpicture}[start chain = A going right]
							\node [sort node, label=1] {1};
							\node [comp node, label=2] {2};
							\node [comp node, label=3] {4};
							\node [comp node, label=4] {5};
							\node [comp node, label=5] {6};
							\node [curs node, label=6] {3};

							\draw [-{Stealth[length=1mm]}, gray] (A-3.290) to[out=270, in=270, looseness=2] (A-4.250);
							\draw [-{Stealth[length=1mm]}, gray] (A-4.290) to[out=270, in=270, looseness=2] (A-5.250);
							\draw [-{Stealth[length=1mm]}, gray] (A-5.290) to[out=270, in=270, looseness=2] (A-6.250);
							\draw [-{Stealth[length=1mm]}, rounded corners=.3cm]   (A-6.300) -- +(0,-.4) -| (A-3.240);
						\end{tikzpicture}
					}
				}
				\hfill
				\floatbox{figure}{\caption{}}{
					\tikzsetnextfilename{Chapter02/insertion-sort-06}
					\scalebox{1.5}{
						\begin{tikzpicture}[start chain = A going right]
							\node [sort node, label=1] {1};
							\node [sort node, label=2] {2};
							\node [sort node, label=3] {3};
							\node [sort node, label=4] {4};
							\node [sort node, label=5] {5};
							\node [sort node, label=6] {6};
						\end{tikzpicture}
					}
				}
			\end{subfloatrow}
		\end{figure}
		\begin{answer}
			\hfill\\
			\begin{figure}[H]
				\floatsetup{valign=t, heightadjust=object, style=plaintop}
				\begin{subfloatrow}[3]
					\centering
					\floatbox{figure}{\caption{}}{
						\tikzsetnextfilename{Chapter02/insertion-sort-answer-01}
						\scalebox{1.5}{
							\begin{tikzpicture}[start chain = A going right]
								\node[comp node, label=1] {31};
								\node[curs node, label=2] {41};
								\node[sort node, label=3] {59};
								\node[sort node, label=4] {26};
								\node[sort node, label=5] {41};
								\node[sort node, label=6] {58};

								\draw [-{Stealth[length=1mm]}] (A-2.300) to[out=270, in=270, looseness=4] (A-2.240);
							\end{tikzpicture}
						}
					}

					\floatbox{figure}{\caption{}}{
						\tikzsetnextfilename{Chapter02/insertion-sort-answer-02}
						\scalebox{1.5}{
							\begin{tikzpicture}[start chain = A going right]
								\node[sort node, label=1] {31};
								\node[comp node, label=2] {41};
								\node[curs node, label=3] {59};
								\node[sort node, label=4] {26};
								\node[sort node, label=5] {41};
								\node[sort node, label=6] {58};

								\draw [-{Stealth[length=1mm]}] (A-3.300) to[out=270, in=270, looseness=4] (A-3.240);
							\end{tikzpicture}
						}
					}

					\floatbox{figure}{\caption{}}{
						\tikzsetnextfilename{Chapter02/insertion-sort-answer-03}
						\scalebox{1.5}{
							\begin{tikzpicture}[start chain = A going right]
								\node[comp node, label=1] {31};
								\node[comp node, label=2] {41};
								\node[comp node, label=3] {59};
								\node[curs node, label=4] {26};
								\node[sort node, label=5] {41};
								\node[sort node, label=6] {58};

								\draw [-{Stealth[length=1mm]}, gray] (A-1.290) to[out=270, in=270, looseness=2] (A-2.250);
								\draw [-{Stealth[length=1mm]}, gray] (A-2.290) to[out=270, in=270, looseness=2] (A-3.250);
								\draw [-{Stealth[length=1mm]}, gray] (A-3.290) to[out=270, in=270, looseness=2] (A-4.250);
								\draw [-{Stealth[length=1mm]}, rounded corners=.3cm]   (A-4.300) -- +(0,-.4) -| (A-1.240);
							\end{tikzpicture}
						}
					}
				\end{subfloatrow}

				\begin{subfloatrow}[3]
					\floatbox{figure}{\caption{}}{
						\tikzsetnextfilename{Chapter02/insertion-sort-answer-04}
						\scalebox{1.5}{
							\begin{tikzpicture}[start chain = A going right]
								\node[sort node, label=1] {26};
								\node[sort node, label=2] {31};
								\node[comp node, label=3] {41};
								\node[comp node, label=4] {59};
								\node[curs node, label=5] {41};
								\node[sort node, label=6] {58};

								\draw [-{Stealth[length=1mm]}, gray] (A-4.290) to[out=270, in=270, looseness=2] (A-5.250);
								\draw [-{Stealth[length=1mm]}]       (A-5.290) to[out=270, in=270, looseness=2] (A-4.250);
							\end{tikzpicture}
						}
					}

					\floatbox{figure}{\caption{}}{
						\tikzsetnextfilename{Chapter02/insertion-sort-answer-05}
						\scalebox{1.5}{
							\begin{tikzpicture}[start chain = A going right]
								\node[sort node, label=1] {26};
								\node[sort node, label=2] {31};
								\node[sort node, label=3] {41};
								\node[comp node, label=4] {41};
								\node[comp node, label=5] {59};
								\node[curs node, label=6] {58};

								\draw [-{Stealth[length=1mm]}, gray] (A-5.290) to[out=270, in=270, looseness=2] (A-6.250);
								\draw [-{Stealth[length=1mm]}]       (A-6.290) to[out=270, in=270, looseness=2] (A-5.250);
							\end{tikzpicture}
						}
					}

					\floatbox{figure}{\caption{}}{
						\tikzsetnextfilename{Chapter02/insertion-sort-answer-06}
						\scalebox{1.5}{
							\begin{tikzpicture}[start chain = A going right]
								\node[sort node, label=1] {26};
								\node[sort node, label=2] {31};
								\node[sort node, label=3] {41};
								\node[sort node, label=4] {41};
								\node[sort node, label=5] {58};
								\node[sort node, label=6] {59};
							\end{tikzpicture}
						}
					}
				\end{subfloatrow}
			\end{figure}
		\end{answer}

		% 2.1-2
		\item Rewrite the \textsc{Insertion-Sort} procedure to sort into nonincreasing instead of nondecreasing order.
		\begin{answer}
			\hfill\\
			\begin{algorithm}[H]
				\SetKwProg{Fn}{def}{\string:}{end}
				\SetKwFunction{FSort}{Insertion-Sort}

				\Fn{\FSort{$A[1 \dots n]$}}{
					\For{$j = n-1\ \KwTo\ 1$}{
						$key \leftarrow A[j]$\;
						$i \leftarrow j + 1$\;
						\While{$i \leq n$ {\bf and} $A[i] < key$}{
							$A[i - 1] \leftarrow A[i]$\;
							$i \leftarrow i + 1$\;
						}
						$A[i - 1] \leftarrow key$\;
					}
				}
			\end{algorithm}
		\end{answer}

		% 2.1-3
		\item Consider the \textbf{\textit{searching problem}}:\\[.5em]
		\textbf{Input:} A sequence of $n$ number $A = \langle a_1, a_2, \dots, a_n \rangle$ and a value $v$.\\[.5em]
		\textbf{Output:} An index $i$ such that $v = A[i]$ or the special value \texttt{NIL} if $v$ does not appear in $A$.\\[.5em]
		Write pseudocode for \textbf{\textit{linear-search}}, which scans through the sequence, looking for $v$. using a loop invariant, prove that your algorithm is correct. Make sure that your loop invariant fulfills the three necessary properties.
		\begin{answer}
			
		\end{answer}

		% 2.1-4
		\item Consider the problem of adding two $n$-bit binary integers, stored in two $n$-element arrays $A$ and $B$, The sum of the two integers should be stored in binary form in an $(n + 1)$-element array $C$. State the problem formally and write pseudocode for adding the two integers.
		\begin{answer}
			
		\end{answer}
	\end{enumerate}

\end{document}