%!TEX root = ../../Algorithms.tex
\documentclass[Chapter02]{subfiles}

\tikzset{
	node distance = 0pt,
	sort node/.style = {
		rectangle,
		draw = black,
		on chain
	},
	curs node/.style = {
		sort node,
		text = white,
		fill = black
	},
	comp node/.style = {
		sort node,
		fill = gray!40
	},
	every label/.style = {font=\tiny}
}



\begin{document}
	\subsection{Insertion Sort}

	\begin{enumerate}[leftmargin=\labelsep]
		% 2.1-1
		\item Using Figure 2.2 as a model, illustrate the operation of \textsc{Insertion-Sort} on the array $A = \langle 31, 41, 59, 26, 41, 58 \rangle$.
		\begin{figure}[H]
			\floatsetup{valign=t, heightadjust=object, style=plaintop}
			\begin{subfloatrow}[3]
				\centering
				\floatbox{figure}{\caption{}}{
					\tikzsetnextfilename{Chapter02/insertion-sort-01}
					\scalebox{1.5}{
						\begin{tikzpicture}[start chain = A going right]
							\node [comp node, label=1] {5};
							\node [curs node, label=2] {2};
							\node [sort node, label=3] {4};
							\node [sort node, label=4] {6};
							\node [sort node, label=5] {1};
							\node [sort node, label=6] {3};

							\draw [-{Stealth[length=1mm]}, gray] (A-1.290) to[out=270, in=270, looseness=2] (A-2.250);
							\draw [-{Stealth[length=1mm]}]       (A-2.300) to[out=270, in=270, looseness=2] (A-1.240);
						\end{tikzpicture}
					}
				}
				\hfill
				\floatbox{figure}{\caption{}}{
					\tikzsetnextfilename{Chapter02/insertion-sort-02}
					\scalebox{1.5}{
						\begin{tikzpicture}[start chain = A going right]
							\node[comp node, label=1] {2};
							\node[comp node, label=2] {5};
							\node[curs node, label=3] {4};
							\node[sort node, label=4] {6};
							\node[sort node, label=5] {1};
							\node[sort node, label=6] {3};

							\draw [-{Stealth[length=1mm]}, gray] (A-2.290) to[out=270, in=270, looseness=2] (A-3.250);
							\draw [-{Stealth[length=1mm]}]       (A-3.300) to[out=270, in=270, looseness=2] (A-2.240);
						\end{tikzpicture}
					}
				}
				\hfill
				\floatbox{figure}{\caption{}}{
					\tikzsetnextfilename{Chapter02/insertion-sort-03}
					\scalebox{1.5}{
						\begin{tikzpicture}[start chain = A going right]
							\node[sort node, label=1] {2};
							\node[sort node, label=2] {4};
							\node[comp node, label=3] {5};
							\node[curs node, label=4] {6};
							\node[sort node, label=5] {1};
							\node[sort node, label=6] {3};

							\draw [-{Stealth[length=1mm]}] (A-4.300) to[out=270, in=270, looseness=4] (A-4.240);
						\end{tikzpicture}
					}
				}
			\end{subfloatrow}

			\begin{subfloatrow}[3]
				\centering
				\floatbox{figure}{\caption{}}{
					\tikzsetnextfilename{Chapter02/insertion-sort-04}
					\scalebox{1.5}{
						\begin{tikzpicture}[start chain = A going right]
							\node [comp node, label=1] {2};
							\node [comp node, label=2] {4};
							\node [comp node, label=3] {5};
							\node [comp node, label=4] {6};
							\node [curs node, label=5] {1};
							\node [sort node, label=6] {3};

							\draw [-{Stealth[length=1mm]}, gray] (A-1.290) to[out=270, in=270, looseness=2] (A-2.250);
							\draw [-{Stealth[length=1mm]}, gray] (A-2.290) to[out=270, in=270, looseness=2] (A-3.250);
							\draw [-{Stealth[length=1mm]}, gray] (A-3.290) to[out=270, in=270, looseness=2] (A-4.250);
							\draw [-{Stealth[length=1mm]}, gray] (A-4.290) to[out=270, in=270, looseness=2] (A-5.250);
							\draw [-{Stealth[length=1mm]}, rounded corners=.3cm]   (A-5.300) -- +(0,-.4) -| (A-1.240);
						\end{tikzpicture}
					}
				}
				\hfill
				\floatbox{figure}{\caption{}}{
					\tikzsetnextfilename{Chapter02/insertion-sort-05}
					\scalebox{1.5}{
						\begin{tikzpicture}[start chain = A going right]
							\node [sort node, label=1] {1};
							\node [comp node, label=2] {2};
							\node [comp node, label=3] {4};
							\node [comp node, label=4] {5};
							\node [comp node, label=5] {6};
							\node [curs node, label=6] {3};

							\draw [-{Stealth[length=1mm]}, gray] (A-3.290) to[out=270, in=270, looseness=2] (A-4.250);
							\draw [-{Stealth[length=1mm]}, gray] (A-4.290) to[out=270, in=270, looseness=2] (A-5.250);
							\draw [-{Stealth[length=1mm]}, gray] (A-5.290) to[out=270, in=270, looseness=2] (A-6.250);
							\draw [-{Stealth[length=1mm]}, rounded corners=.3cm]   (A-6.300) -- +(0,-.4) -| (A-3.240);
						\end{tikzpicture}
					}
				}
				\hfill
				\floatbox{figure}{\caption{}}{
					\tikzsetnextfilename{Chapter02/insertion-sort-06}
					\scalebox{1.5}{
						\begin{tikzpicture}[start chain = A going right]
							\node [sort node, label=1] {1};
							\node [sort node, label=2] {2};
							\node [sort node, label=3] {3};
							\node [sort node, label=4] {4};
							\node [sort node, label=5] {5};
							\node [sort node, label=6] {6};
						\end{tikzpicture}
					}
				}
			\end{subfloatrow}
		\end{figure}
		\begin{answer}
			\hfill\\
			\begin{figure}[H]
				\floatsetup{valign=t, heightadjust=object, style=plaintop}
				\begin{subfloatrow}[3]
					\centering
					\floatbox{figure}{\caption{}}{
						\tikzsetnextfilename{Chapter02/insertion-sort-answer-01}
						\scalebox{1.5}{
							\begin{tikzpicture}[start chain = A going right]
								\node[comp node, label=1] {31};
								\node[curs node, label=2] {41};
								\node[sort node, label=3] {59};
								\node[sort node, label=4] {26};
								\node[sort node, label=5] {41};
								\node[sort node, label=6] {58};

								\draw [-{Stealth[length=1mm]}] (A-2.300) to[out=270, in=270, looseness=4] (A-2.240);
							\end{tikzpicture}
						}
					}

					\floatbox{figure}{\caption{}}{
						\tikzsetnextfilename{Chapter02/insertion-sort-answer-02}
						\scalebox{1.5}{
							\begin{tikzpicture}[start chain = A going right]
								\node[sort node, label=1] {31};
								\node[comp node, label=2] {41};
								\node[curs node, label=3] {59};
								\node[sort node, label=4] {26};
								\node[sort node, label=5] {41};
								\node[sort node, label=6] {58};

								\draw [-{Stealth[length=1mm]}] (A-3.300) to[out=270, in=270, looseness=4] (A-3.240);
							\end{tikzpicture}
						}
					}

					\floatbox{figure}{\caption{}}{
						\tikzsetnextfilename{Chapter02/insertion-sort-answer-03}
						\scalebox{1.5}{
							\begin{tikzpicture}[start chain = A going right]
								\node[comp node, label=1] {31};
								\node[comp node, label=2] {41};
								\node[comp node, label=3] {59};
								\node[curs node, label=4] {26};
								\node[sort node, label=5] {41};
								\node[sort node, label=6] {58};

								\draw [-{Stealth[length=1mm]}, gray] (A-1.290) to[out=270, in=270, looseness=2] (A-2.250);
								\draw [-{Stealth[length=1mm]}, gray] (A-2.290) to[out=270, in=270, looseness=2] (A-3.250);
								\draw [-{Stealth[length=1mm]}, gray] (A-3.290) to[out=270, in=270, looseness=2] (A-4.250);
								\draw [-{Stealth[length=1mm]}, rounded corners=.3cm]   (A-4.300) -- +(0,-.4) -| (A-1.240);
							\end{tikzpicture}
						}
					}
				\end{subfloatrow}

				\begin{subfloatrow}[3]
					\floatbox{figure}{\caption{}}{
						\tikzsetnextfilename{Chapter02/insertion-sort-answer-04}
						\scalebox{1.5}{
							\begin{tikzpicture}[start chain = A going right]
								\node[sort node, label=1] {26};
								\node[sort node, label=2] {31};
								\node[comp node, label=3] {41};
								\node[comp node, label=4] {59};
								\node[curs node, label=5] {41};
								\node[sort node, label=6] {58};

								\draw [-{Stealth[length=1mm]}, gray] (A-4.290) to[out=270, in=270, looseness=2] (A-5.250);
								\draw [-{Stealth[length=1mm]}]       (A-5.290) to[out=270, in=270, looseness=2] (A-4.250);
							\end{tikzpicture}
						}
					}

					\floatbox{figure}{\caption{}}{
						\tikzsetnextfilename{Chapter02/insertion-sort-answer-05}
						\scalebox{1.5}{
							\begin{tikzpicture}[start chain = A going right]
								\node[sort node, label=1] {26};
								\node[sort node, label=2] {31};
								\node[sort node, label=3] {41};
								\node[comp node, label=4] {41};
								\node[comp node, label=5] {59};
								\node[curs node, label=6] {58};

								\draw [-{Stealth[length=1mm]}, gray] (A-5.290) to[out=270, in=270, looseness=2] (A-6.250);
								\draw [-{Stealth[length=1mm]}]       (A-6.290) to[out=270, in=270, looseness=2] (A-5.250);
							\end{tikzpicture}
						}
					}

					\floatbox{figure}{\caption{}}{
						\tikzsetnextfilename{Chapter02/insertion-sort-answer-06}
						\scalebox{1.5}{
							\begin{tikzpicture}[start chain = A going right]
								\node[sort node, label=1] {26};
								\node[sort node, label=2] {31};
								\node[sort node, label=3] {41};
								\node[sort node, label=4] {41};
								\node[sort node, label=5] {58};
								\node[sort node, label=6] {59};
							\end{tikzpicture}
						}
					}
				\end{subfloatrow}
			\end{figure}
		\end{answer}

		% 2.1-2
		\item Rewrite the \textsc{Insertion-Sort} procedure to sort into nonincreasing instead of nondecreasing order.
		\begin{answer}
			We simply reverse everything: start at the end of the array and go backwards instead of at the start and forwards, and search forwards for a larger element instead of searching backwards for a smaller element:\\
			\begin{algorithm}[H]
				\SetKwProg{Fn}{def}{\string:}{end}
				\SetKwFunction{FSort}{Insertion-Sort}
				\SetKw{KwAnd}{and}

				\Fn{\FSort{$A[1 \dots n]$}}{
					\For{$j \leftarrow n-1\ \KwTo\ 1$}{
						$key \leftarrow A[j]$\;
						$i \leftarrow j + 1$\;
						\While{$i \leq n\ \KwAnd\ A[i] < key$}{
							$A[i - 1] \leftarrow A[i]$\;
							$i \leftarrow i + 1$\;
						}
						$A[i - 1] \leftarrow key$\;
					}
				}
			\end{algorithm}
		\end{answer}

		% 2.1-3
		\item \label{exer:ch02-linear-search} Consider the \textbf{\textit{searching problem}}:
		\begin{description}
			\item[Input:] A sequence of $n$ numbers $A = \langle a_1, a_2, \dots, a_n \rangle$ and a value $v$.
			\item[Output:] An index $i$ such that $v = A[i]$ or the special value \texttt{NIL} if $v$ does not appear in $A$.
		\end{description}
		Write pseudocode for \textbf{\textit{linear-search}}, which scans through the sequence, looking for $v$. using a loop invariant, prove that your algorithm is correct. Make sure that your loop invariant fulfills the three necessary properties.
		\begin{answer}
			\hfill\\
			\begin{algorithm}[H]
				\SetKwProg{Fn}{def}{\string:}{end}
				\SetKwFunction{FSearch}{Linear-Search}

				\Fn{\FSearch{$A[1 \dots n],\ v$}}{
					\For{$i \leftarrow 1\ \KwTo\ n$}{
						\If{$A[i] = v$}{
							\Return{i}
						}
					}
					\Return{\tt NIL}
				}
			\end{algorithm}
			We use the following loop invariant to show that the algorithm returns \texttt{NIL} if and only if $v$ is not present in $A$:

			\begin{addmargin}[2em]{2em}
				At the start of each iteration of the \textbf{for} loop of lines 2-6, the sub-array $A[1 \dots i - 1]$ does not contain $v$.
			\end{addmargin}

			We then check that the loop invariant holds.

			\begin{description}
				\item[Initialization:] This is trivially true for $i = 1$, as the sub-array $A[1 \dots 0]$ is empty.

				\item[Maintenance:] We know that none of the elements in $A[1 \dots i - 1]$ are $v$, and the only additional element in $A[1 \dots i]$ is $A[i]$, which cannot be $v$, otherwise we would have exited from the loop with the \textbf{if} statement on lines 3-5. Therefore the sub-array $A[1 \dots i]$ does not contain $v$.

				\item[Termination:] The loop terminates when $i = n + 1$, so $A[1 \dots i - 1] = A[1 \dots n]$ does not contain $v$. Therefore, we can safely return \texttt{NIL}.
			\end{description}
			
			Since \texttt{NIL} is returned if and only if $v$ is not in $A$, then an index $i$ must be returned if $v$ is in $A$ - this can only happen when $A[i] = v$ (as seen in lines 3-5). Therefore the algorithm is correct.
		\end{answer}

		% 2.1-4
		\item Consider the problem of adding two $n$-bit binary integers, stored in two $n$-element arrays $A$ and $B$, The sum of the two integers should be stored in binary form in an $(n + 1)$-element array $C$. State the problem formally and write pseudocode for adding the two integers.
		\begin{answer}
			We define the \textbf{\textit{binary addition problem}}:
			\begin{description}
				\item[Input:] Two $n$-element sequences $A = \langle a_1, a_2, \dots, a_n \rangle$ and $B = \langle b_1, b_2, \dots, b_n \rangle$ which represent binary numbers, $a_i,b_i \in \{0, 1\}$ for all $1 \leq i \leq n$.

				\item[Output:] An $(n + 1)$-element sequence $C = \langle c_1, c_2, \dots, c_{n + 1} \rangle$ which represents the binary sum of the two binary numbers given above, i.e. $c_i \in \{0, 1\}$ for all $1 \leq i \leq n + 1$
				\[
					\sum_{i = 1}^{n + 1} c_i2^{i - 1} = \qty(\sum_{i = 1}^{n} a_i2^{i - 1}) + \qty(\sum_{i = 1}^{n} b_i2^{i - 1}).
				\]
			\end{description}

			We then note a key fact about binary addition, which lets us ``carry'' in a similar way to normal base 10 addition:
			\begin{align}
				1\cdot2^i + 1\cdot2^i &= 2\cdot2^i \nonumber\\
				                      &= 1\cdot2^{i + 1} + 0\cdot2^i. \label{eq:ch02-binary-add-carry}
			\end{align}

			With this in mind, we can write pseudocode for an algorithm \textsc{Binary-Add} to solve the \textbf{\textit{binary addition problem}}:

			\begin{algorithm}[H]
				\SetKwProg{Fn}{def}{\string:}{end}
				\SetKwFunction{FBin}{Binary-Add}

				\Fn{\FBin{$A[1 \dots n], B[1 \dots n]$}}{
					\tcp{We only carry from a previous addition, which does not exist at the beginning of the algorithm}
					$carry \leftarrow 0$\;
					\For{$i \leftarrow 1\ \KwTo\ n$}{
						$C[i] \leftarrow A[i] + B[i] + carry$\;
						\eIf{$C[i] > 1$}{
							$C[i] \leftarrow C[i] - 2$\;
							$carry \leftarrow 1$\;
						}{
							$carry \leftarrow 0$\;
						}
					}
					$C[n + 1] \leftarrow carry$\;
					\Return{$C[1 \dots (n + 1)]$}
				}
			\end{algorithm}

			We then use the following loop invariant to show the correctness of \textsc{Binary-Add}:

			\begin{addmargin}[2em]{2em}
				At the start of each iteration of the \textbf{for} loop of lines 3-11, the values $c_j \in \{0,1\}$ for $1 \leq j \leq i - 1$ and
				\[
					\sum_{j = 1}^{i - 1} c_j2^{j - 1} + (carry_{(i - 1)})2^{i-1} = \qty(\sum_{j = 1}^{i - 1} a_j 2^{j - 1}) + \qty(\sum_{j = 1}^{i - 1} b_j 2^{j - 1}).
				\]
			\end{addmargin}

			\begin{description}
				\item[Initialization:] This is trivially true, as $i - 1 = 0$, so we don't have to worry about any elements of $C$ yet. As well, the sum evaluates to $0 + (carry_1)2^0 = 0 + 0$, which is true, since $carry_1 = 0$.

				\item[Maintenance:] Observe that $carry \in \{0,1\}$ for the entire algorithm. Since $a_i, b_i, carry_{(i - 1)} \leq 1$, then $a_i + b_i + carry_{(i - 1)} \leq 3$. Then we have two cases:
				\begin{description}
					\item[Case 1:] $c_i = a_i + b_i + carry_{(i - 1)} \leq 1$. Then $c_i \in \{0, 1\}$, the next $carry_i = 0$, and
					\begin{align*}
						\sum_{j = 1}^{i} c_j2^{j - 1} + (carry_i) \cdot 2^{i} &= \sum_{j = 1}^{i - 1} c_j2^{j - 1} + c_i2^{i - 1}\\
							&= \sum_{j = 1}^{i - 1} c_j2^{j - 1} + \qty(a_i + b_i + (carry_{(i - 1)}))2^{i - 1}\\
							&= \qty(\sum_{j = 1}^{i - 1} c_j2^{j - 1} + (carry_{(i - 1)})2^{i - 1}) + (a_i + b_i)2^{i - 1}\\
							&= \qty(\sum_{j = 1}^{i - 1} a_j 2^{j - 1}) + \qty(\sum_{j = 1}^{i - 1} b_j 2^{j - 1}) + (a_i + b_i)2^{i - 1}\\
							&= \qty(\sum_{j = 1}^{i - 1} a_j 2^{j - 1} + a_i2^{i - 1}) + \qty(\sum_{j = 1}^{i - 1} b_j 2^{j - 1} + b_i2^{i-1})\\
							&= \qty(\sum_{j = 1}^i a_j 2^{j - 1}) + \qty(\sum_{j = 1}^i b_j 2^{j - 1}).
					\end{align*}

					\item[Case 2:] $2 \leq a_i + b_i + carry_{(i - 1)} \leq 3$. Then $c_i = a_i + b_i + carry_{(i - 1)} - 2 \in \{0, 1\}$, the next $carry_i = 1$, and
					\begin{align*}
						\sum_{j = 1}^{i} c_j2^{j - 1} + (carry_i) \cdot 2^{i} &= \sum_{j = 1}^{i - 1} c_j2^{j - 1} + c_i2^{i - 1} + 2^i\\
							&= \sum_{j = 1}^{i - 1} c_j2^{j - 1} + (a_i + b_i + carry_{(i - 1)} - 2)2^{i - 1} + 2^i\\
							&= \qty(\sum_{j = 1}^{i - 1} c_j2^{j - 1} + carry_{(i - 1)}2^{i - 1}) + (a_i + b_i) 2^{i - 1} - \cancel{2^i + 2^i}\\
							&= \qty(\sum_{j = 1}^{i - 1} a_j 2^{j - 1}) + \qty(\sum_{j = 1}^{i - 1} b_j 2^{j - 1}) + (a_i + b_i) 2^{i - 1}\\
							&= \qty(\sum_{j = 1}^{i - 1} a_j 2^{j - 1} + a_i 2^{i - 1}) + \qty(\sum_{j = 1}^{i - 1} b_j 2^{j - 1} + b_i 2^{i - 1})\\
							&= \qty(\sum_{j = 1}^i a_j 2^{j - 1}) + \qty(\sum_{j = 1}^i b_j 2^{j - 1}).
					\end{align*}
				\end{description}

				\item[Termination:] The loop terminates when $i = n + 1$, so $c_j \in \{0,1\}$ for all $1 \leq j \leq i - 1 = n$. As well, $c_{n + 1} = carry_n \in \{0, 1\}$. Then we have
				\begin{align*}
					\sum_{j = 1}^{n + 1} c_j2^{j - 1} &= \sum_{j = 1}^n c_j2^{j - 1} + c_{n + 1}2^n\\
						&= \sum_{j = 1}^{i - 1} c_j2^{j - 1} + carry_{(i - 1)} 2^{i - 1}\\
					\shortintertext{(Since the algorithm terminates by assigning $C[n + 1] \leftarrow carry$)}
						&= \qty(\sum_{j = 1}^{i - 1} a_j 2^{j - 1}) + \qty(\sum_{j = 1}^{i - 1} b_j 2^{j - 1})\\
						&= \qty(\sum_{j = 1}^n a_j 2^{j - 1}) + \qty(\sum_{j = 1}^n b_j 2^{j - 1}).
				\end{align*}
				Therefore, our algorithm is correct.
			\end{description}
		\end{answer}
	\end{enumerate}

\end{document}