%!TEX root = ../../Algorithms.tex
\documentclass[AppendixA]{subfiles}

\begin{document}
	\subsection{Summation formulas and properties}

	\begin{enumerate}
		% A.1-1
		\item Find a simple formula for
		\[
			\sum_{k = 1}^n (2k - 1).
		\]
		\begin{answer}
			By linearity, we have
			\begin{align*}
				\sum_{k = 1}^n (2k - 1) &= 2 \sum_{k = 1}^n k - \sum_{k = 1}^n 1\\
					&= 2 \qty(\frac{n(n+1)}{2}) - n\\
					&= n(n+1) - n\\
					&= n(n+1 - 1)\\
					&= n^2.
			\end{align*}
		\end{answer}

		% A.1-2
		\item $\bigstar$ Show that
		\[
			\sum_{k = 1}^n \frac{1}{2k - 1} = \ln(\sqrt{n}) + \bigO(1)
		\]
		by manipulating the harmonic series.
		\begin{answer}
			Note that $2k - 1$ for $k = 1 \dots n$ assumes only all of the odd numbers between $1$ and $2n$. This observation, combined with the observation that $2k$ for $k = 1 \dots n$ assumes only all of the even numbers between $1$ and $2n$ leads us to the following:
			\begin{align*}
				\sum_{k = 1}^n \frac{1}{2k - 1} &= \sum_{k = 1}^n \frac{1}{2k - 1} + \sum_{k = 1}^n \frac{1}{2k} - \sum_{k = 1}^n \frac{1}{2k}\\
					&= \sum_{k = 1}^{2n} \frac{1}{k} - \frac{1}{2}\sum_{k = 1}^n \frac{1}{k}\\
					&= \ln(n) + \bigO(1) - \frac{1}{2}\qty(\ln(n) + \bigO(1))\\
					&= \frac{1}{2} \ln(n) + \bigO(1)\\
					&= \ln(\sqrt{n}) + \bigO(1).
			\end{align*}
		\end{answer}
		
		% A.1-3
		\item Show that
		\[
			\sum_{k = 0}^\infty k^2 x^k = \frac{x(1 + x)}{(1 - x)^3}
		\]
		for $|x| < 1$.
		\begin{answer}
			Since $|x| < 1$, we know that the sum
			\[
				\sum_{k = 0}^\infty kx^k = \frac{x}{(1-x)^2}
			\]
			converges. Then, we take the derivative of each side, and find
			\begin{align*}
				\sum_{k = 0}^\infty k^2x^{k - 1} &= \dv{}{x} \qty[\frac{x}{(1-x)^2}]\\
					&= \frac{1(1-x)^2 - x(-2(1 - x))}{(1-x)^4}\\
					&= \frac{(1 - 2x + x^2 )+ (2x - 2x^2)}{(1-x)^4}\\
					&= \frac{1 - x^2}{(1-x)^4}\\
					&= \frac{(1 - x)(1 + x)}{(1-x)^4}\\
					&= \frac{1 + x}{(1-x)^3}.
			\end{align*}
			Then we have
			\begin{align*}
				\sum_{k = 0}^\infty k^2 x^k &= x \sum_{k = 0}^\infty k^2x^{k - 1}\\
					&= x \qty(\frac{1 + x}{(1-x)^3})\\
					&= \frac{x(1 + x)}{(1-x)^3}.
			\end{align*}
		\end{answer}
		
		% A.1-4
		\item $\bigstar$ Show that
		\[
			\sum_{k = 0}^\infty \frac{k - 1}{2^k} = 0.
		\]
		\begin{answer}
			As the sum of convergent known geometric-like series, the given series converges. Then we have
			\begin{align*}
				\sum_{k = 0}^\infty \frac{k - 1}{2^k} &= \sum_{k = 0}^\infty \frac{k}{2^k} - \sum_{k = 0}^\infty \frac{1}{2^k}\\
					&= \sum_{k = 0}^\infty k\qty(\frac{1}{2})^k - \sum_{k = 0}^\infty \qty(\frac{1}{2})^k\\
					&= \frac{\frac{1}{2}}{\qty(1 - \frac{1}{2})^2} - \frac{1}{1 - \frac{1}{2}}\\
					&= 2 - 2\\
					&= 0.
			\end{align*}
		\end{answer}
		
		% A.1-5
		\item $\bigstar$ Evaluate the sum
		\[
			\sum_{k = 1}^\infty (2k + 1)x^{2k}
		\]
		for $|x| < 1$.
		\begin{answer}
			Note that for $|x| < 1$, we also have $|x^2| < 1$. Then we have
			\begin{align*}
				\sum_{k = 1}^\infty x^{2k+1} &= x\sum_{k = 1}^\infty x^{2k}\\
					&= x\sum_{k = 1}^\infty \qty(x^2)^k\\
					&= x\qty(\sum_{k = 0}^\infty \qty(x^2)^k - 1)\\
					&= x\qty(\frac{1}{1 - x^2} - 1)\\
					&= x\frac{1 - (1 - x^2)}{1 - x^2}\\
					&= \frac{x^3}{1 - x^2}.
			\end{align*}
			Then, we can take derivatives of both sides, and get
			\begin{align*}
				\sum_{k = 1}^\infty (2k+1)x^{2k} &= \dv{}{x} \qty[\frac{x^3}{1 - x^2}]\\
					&= \frac{3x^2(1 - x^2) - x^3(-2x)}{(1 - x^2)^2}\\
					&= \frac{3x^2 - 3x^4 + 2x^4}{(1 - x^2)^2}\\
					&= \frac{x^2\qty(3 - x^2)}{(1 - x^2)^2}.
			\end{align*}
		\end{answer}
		
		% A.1-6
		\item Prove that
		\[
			\sum_{k = 1}^n \bigO(f_k(i)) = \bigO\qty(\sum_{k = 1}^n f_k(i))
		\]
		by using the linearity property of summations.
		\begin{answer}
			For some sequence $\{c_k\}_{k \leq n}$, we have
			\begin{align*}
				\sum_{k = 1}^n \bigO(f_k(i)) &\leq \sum_{k = 1}^n c_kf_k(i)\\
					&\leq \sum_{k = 1}^n \max_{k \leq n}\{c_k\}f_k(i)\\
					&= \max_{k \leq n}\{c_k\} \sum_{k = 1}^n f_k(i),
			\end{align*}
			so $ \displaystyle \sum_{k = 1}^n \bigO(f_k(i)) = \bigO\qty(\sum_{k = 1}^n f_k(i))$.
		\end{answer}
		
		% A.1-7
		\item Evaluate the product
		\[
			\prod_{k = 1}^n 2 \cdot 4^k.
		\]
		\begin{answer}
			As a finite product, we can ``reorder'' the factors using commutativity of multiplication so that all of the 2 factors come before all of the $4^k$ factors. This gives an identity similar to linearity of the sum:
			\begin{align*}
				\prod_{k = 1}^n 2 \cdot 4^k &= \prod_{k = 1}^n 2 \cdot \prod_{k = 1}^n 4^k\\
					&= 2^n \prod_{k = 1}^n 2^{2k}\\
					&= 2^n \exp(\ln(\prod_{k = 1}^n 2^{2k}))\\
					&= 2^n \exp(\sum_{k = 1}^n \ln(2^{2k}))\\
					&= 2^n \exp(\sum_{k = 1}^n 2k\ln(2))\\
					&= 2^n \exp(2\ln(2)\sum_{k = 1}^n k)\\
					&= 2^n \exp(\ln(2)n(n + 1))\\
					&= 2^n \cdot 2^{n(n + 1)}\\
					&= 2^{n + n(n+1)}\\
					&= 2^{n(n+2)}.
			\end{align*}
		\end{answer}
		
		% A.1-8
		\item $\bigstar$ Evaluate the product
		\[
			\prod_{k = 2}^n \qty(1 - \frac{1}{k^2}).
		\]
		\begin{answer}
			We have
			\begin{align*}
				\prod_{k = 2}^n \qty(1 - \frac{1}{k^2}) &= \prod_{k = 2}^n \frac{k^2 - 1}{k^2}\\
					&= \prod_{k = 2}^n \frac{(k - 1)(k + 1)}{k^2}.
			\end{align*}
			This is a sort of ``telescoping'' product, which cancels out in a similar way to a telescoping sum. We can observe this by writing out a few of the first factors and a few of the last factors:
			\begin{align*}
				\prod_{k = 2}^n \frac{(k - 1)(k + 1)}{k^2} &= \qty(\frac{1(3)}{2^2})\qty(\frac{2(4)}{3^2})\qty(\frac{3(5)}{4^2}) \dots \qty(\frac{(n - 3)(n - 1)}{(n - 2)^2})\qty(\frac{(n - 2)n}{(n - 1)^2})\qty(\frac{(n - 1)(n + 1)}{n^2})\\
					&= \qty(\frac{1(\cancel{3})}{2^{\cancel{2}}})\qty(\frac{\cancel{2}(\cancel{4})}{\cancel{3^2}})\qty(\frac{\cancel{3}(\cancel{5})}{\cancel{4^2}}) \dots \qty(\frac{\cancel{(n - 3)}\cancel{(n - 1)}}{\cancel{(n - 2)^2}})\qty(\frac{\cancel{(n - 2)}\cancel{n}}{\cancel{(n - 1)^2}})\qty(\frac{\cancel{(n - 1)}(n + 1)}{n^{\cancel{2}}})\\
					&= \frac{n+1}{2n}.
			\end{align*}
		\end{answer}
	\end{enumerate}

\end{document}