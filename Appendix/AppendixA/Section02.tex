%!TEX root = ../../Algorithms.tex
\documentclass[AppendixA]{subfiles}

\begin{document}
	\subsection{Bounding Summations}

	\begin{enumerate}[leftmargin=\labelsep]
		% A.2-1
		\item Show that
		\[
			\sum_{k = 1}^n \frac{1}{k}^2
		\]
		is bounded above by a constant.
		\begin{answer}
			
		\end{answer}

		% A.2-2
		\item Find an aymptotic upper bound on the summation
		\[
			\sum_{k = 0}^{\lfloor \lg n \rfloor} \left\lceil \frac{n}{2^k} \right\rceil.
		\]
		\begin{answer}
			
		\end{answer}
		
		% A.2-3
		\item Show that the $n$-th harmonic number is $\Omega(\lg n)$ by splitting the summation.
		\begin{answer}
			
		\end{answer}
		
		% A.2-4
		\item Approximate
		\[
			\sum_{k = 1}^n k^3
		\]
		with an integral.
		\begin{answer}
			
		\end{answer}
		
		% A.2-5
		\item Why didn't we use the integral approximation (A.12) directly on
		\[
			\sum_{k = 1}^n \frac{1}{k}
		\]
		to obtain an upper bound on the $n$-th harmonic number?
		\begin{answer}
			
		\end{answer}
		
	\end{enumerate}
\end{document}